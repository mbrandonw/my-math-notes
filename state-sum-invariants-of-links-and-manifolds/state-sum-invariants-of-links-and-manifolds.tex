

\input{"/Users/brandonwilliams/Documents/LaTeX Includes/notespreamble.tex"}
\input{"/Users/brandonwilliams/Documents/LaTeX Includes/extrapackages.tex"}
\input{"/Users/brandonwilliams/Documents/LaTeX Includes/extracommands.tex"}


\begin{document}


\title{\Large State Sum Invariants of Links and Manifolds}
\author{Brandon Williams \\ \texttt{mbw@math.sunysb.edu}}
\maketitle


\tableofcontents




\newpage
\section{Introduction}
\label{sec Introduction}




\newpage
\section{Classical Recoupling Theory}
\label{sec Classical Recoupling Theory}


In this section we will study the theory behind the decomposition of tensor products of irreducible representations of an algebra $U\mathfrak{sl}_2$ into direct sums of irreducibles. This has physical meaning in the theory of angular momentum, but we do not discuss this aspect. There is a diagrammatic aspect to this theory, which will be useful for calculations and for when we discuss the quantum analogue of these ideas. The diagrammatical approach was first developed by Penrose in his \emph{spin networks}. 


\subsection{The Algebra $U\mathfrak{sl}_2$ and its Representations}
\label{subsec The Lie Algebra sl2 and its Representations}

Let us review some basic facts about the Lie algebra $\mathfrak{sl}_2$ and the representation theory of its universal enveloping algebra $U\mathfrak{sl}_2$. It is the $\mathbb C$-vector space generated by elements $E,F,H$ and subject to the relations
\[ [H,E] = 2E \ \ \ \ \ [H,F]=-2F \ \ \ \ \ [E,F] = H \]
Since the representation theory of $\mathfrak{sl}_2$ is identical to that of its universal enveloping algebra we will actually concern ourselves with this algebra most of the time. Recall that this is the associative $\mathbb C$-algebra generated by $E,F,H$ and subject to the relations
\[ HE - EH = 2E \ \ \ \ \ HF - FH = -2F \ \ \ \ \ EF - FE = H \]

It is useful to first know that $U\mathfrak{sl}_2$'s representation theory is semi-simple; that is, every representation decomposes as a direct sum of irreducibles.

\begin{thm}
Every representation of $U\mathfrak{sl}_2$ is completely reducible, i.e. is a direct sum of irreducibles.
\end{thm}
\begin{proof}
The easiest way to see this is to observe that the representation theory of $U\mathfrak{sl}_2$ is identical to the representation theory of $SU(2)$ (as a Lie group). Since $SU(2)$ is a compact Lie group we can equip it with a Haar measure so that every representation of $SU(2)$ is actually unitary, and every unitary representation is completely reducible.
\end{proof}

The most obvious representation of $U\mathfrak{sl}_2$ is the trivial 1-dimensional representation $V_0 = \mathbb C$, where $E,F,H$ all act as multiplication by 0, and it is clearly irreducible. There is a more interesting representation of $U\mathfrak{sl}_2$ on a 2-dimensional vector space $V_1 = \mathbb Ce_0 \oplus \mathbb Ce_1$ by realizing that we can identify $\mathfrak{sl}_2$ with $T_I \SL(2)$, the traceless complex matrices. So we let $E,F,H$ act as multiplication by the following matrices
\[ E = \begin{pmatrix} 0 & 1 \\ 0 & 0 \end{pmatrix} \ \ \ \ \ F = \begin{pmatrix} 0 & 0 \\ 1 & 0 \end{pmatrix} \ \ \ \ \ H = \begin{pmatrix} 1 & 0 \\ 0 & -1 \end{pmatrix} \]
We can easily compute the relations
\[
\begin{array}{rclcrcl}
 E e_0 &=& 0   & & E e_1 &=& e_0 \\
 F e_0 &=& e_1 & & F e_1 &=& 0   \\
 H e_0 &=& e_0 & & H e_1 &=& -e_1 
\end{array}
\]
This representation is clearly irreducible, for any non-zero sub-representation must contain one of $e_0$ or $e_1$ either of which generates all of $V_1$. From this representation, often called the fundamental representation, we can construct many representations by taking tensor powers. We extend the action of $E,F,H$ on $V_1$ to the tensor power $V_1^{\otimes k}$ by forcing $E,F,H$ to act like a derivation
\[ E v_{\epsilon_1} \otimes \cdots \otimes v_{\epsilon_k} = \sum_{i=1}^k v_{\epsilon_1} \otimes \cdots \otimes E v_{\epsilon_i} \otimes \cdots \otimes v_{\epsilon_k} \ \ \ \ , \epsilon_i = 0 \text{ or } 1 \]
and so on. For $k \geq 2$ these representations will \emph{never} be irreducible. For example, in $V_1 \otimes V_1$ there is a sub-representation isomorphic to the trivial representation by taking the subspace generated by $e_0 \otimes e_1 - e_1 \otimes e_0$. One can easily check that $E,F,H$ act trivially on this vector, hence it generates a sub-representation isomorphic to $V_0$.

Let us define a new family of representations $V_k := S^k(V_1)$ by taking the $k$-th symmetric power of $V_1$. The action of $E,F,H$ on the $k$-th tensor power of $V_1$ descends to the quotient by symmetric tensors, so we get a well-defined action of $E,F,H$ on $V_k$. Clearly $V_k$ is of dimension $(k+1)$ over $\mathbb C$. We can think of elements of $V_k$ as homogeneous, symmetric polynomials in $e_0,e_1$ of degree $k$, hence it is linearly spanned by the monomials $e_i := e_0^i e_1^{k-i}$, $i=0,\ldots,k$. We can easily write down the actions of $E,F,H$ on this basis:
\begin{equation}
\label{eqn sl2 action on Vk}
\begin{array}{rclcrcl}
	E e_i &=& (k-i) e_{i+1} \\
	F e_i &=& i e_{i-1} \\
	H e_i &=& (2i - k) e_i 
\end{array}
\end{equation}

All of the representations $V_k$ are irreducible of dimension $(k+1)$. Any non-zero sub-representation of $V_k$ must contain one of the vectors $e_i$, and by applying $E$ and $F$ repeatedly we can obtain the other $e_j$'s (see \cref{eqn sl2 action on Vk}), hence $V_k$ is irreducible. 

For any representation $V$ of $U\mathfrak{sl}_2$, a \textbf{weight vector} of $V$ is an eigenvector $v$ of the action of $H$, and $v$'s eigenvalue is called its \textbf{weight}. For example, the representation $V_k$ has each $e_i$ as a weight vector of weight $(2i-k)$ (see \cref{eqn sl2 action on Vk}). For a weight $\lambda \in \mathbb C$ of $V$, let $V[\lambda] \subseteq V$ denote the subspace of weight vectors of weight $\lambda$. A weight $\lambda$ for $V$ such that $\re \lambda \geq \re \lambda'$ for every other weight $\lambda'$ of $V$ is called a \textbf{highest weight} of $V$, and the vectors in $V[\lambda]$ are called the \textbf{highest weight vectors}.

\begin{lem}
For any representation $V$ of $U\mathfrak{sl}_2$ and weight $\lambda$ we have
\[ \begin{array}{c}
	EV[\lambda] \subset V[\lambda+2] \\
	FV[\lambda] \subset V[\lambda-2] 
\end{array} \]
\end{lem}
\begin{proof}
Let $v \in V[\lambda]$, then
\[ HEv = EHv + 2Ev = \lambda Ev + 2Ev = (\lambda + 2)Ev \]
hence $Ev \in V[\lambda+2]$. A similar argument works for $F$.
\end{proof}

\begin{lem}
Let $W$ be any finite dimensional representation of $U\mathfrak{sl}_2$ with a weight vector $v$ of weight $\lambda$ such that $Ev=0$. Then $\lambda$ is a non-negative integer, and $W$ contains a sub-representation isomorphic to the irreducible representation $V_\lambda$ such that the highest weight vector $e_0^\lambda$ corresponds to $v$.
\end{lem}
\begin{proof}
%Although \emph{a priori} the weight $\lambda$ could be any complex number, it turns out that messing with the relations of $U\mathfrak{sl}_2$ shows that $\lambda$ is actually a non-negative integer. Let us define a sequence of vectors by $v_0 = v$ and $v_r = F^r v$ for $r \geq 1$. From the above lemma we know that $v_r$ has weight $\lambda-2r$. We have the relation
%\[ Ev_1 = EFv = FEv + Hv = \lambda v \]
%Let us assume, by induction, that we have constants $\gamma_r$ such that $Ev_r=\gamma_rv_{r-1}$, hence $\gamma_0=0$ and $\gamma_1=\lambda$. Then we have
%\[ Ev_r = EFv_{r-1} = FEv_{r-1} + Hv_{r-1} = \gamma_{r-1}Fv_{r-2} + (\lambda-2(r-1))v_{r-1} = (\gamma_{r-1} + \lambda - 2(r-1)) v_{r-1} \]
%Hence $\gamma_r = \gamma_{r-1} + \lambda - 2(r-1)$, which implies that $\gamma_r = r(\lambda - r + 1)$. Now, since each $v_r$ has a different eigenvalue we have that they are linearly independent, hence $v_r = 0$ for large enough $r$. So, take $r$ large enough so that $v_r=0$ and $v_{r-1} \neq 0$. This implies that
%\[ 0 = Ev_r = \gamma_r v_{r-1} = r(\lambda-r+1) v_{r-1} \]
%hence $\lambda = r - 1$, which is a non-negative integer since $r \geq 1$. Consider the sub-representation of $W$ generated by $v$. It is spanned by the vectors $v,v_1,\ldots,v_{r-1}$, which is isomorphic to the irreducible representation $V_r$. In fact, the highest weight vector $e_0^r$ in $V_r$ corresponds to $v \in W$ under this isomorphism.
\end{proof}

An easy corollary of this lemma is the following classification of $U\mathfrak{sl}_2$-representations.

\begin{cor}
\label{thm Classification of sl2 Representations}
Every irreducible representation of $U\mathfrak{sl}_2$ is isomorphic to some $V_k$ ($k \geq 0$).
\end{cor}
\begin{proof}
%Let $W$ be a finite dimensional representation of $U\mathfrak{sl}_2$, and $w$ a non-zero eigenvector of $H$ of weight $\lambda$. We can find an $r \geq 1$ such that $E^{r+1}w=0$ yet $E^r w \neq 0$. Then the vector $v = E^r w$ satisfies the hypotheses of the previous lemma, and $v$ has weight $\lambda-2r$. By the previous lemma we have that $W$ contains a sub-representation isomorphic to $V_{\lambda-2r}$
\end{proof}



\subsection{A Diagrammatic Calculus for the Representation Theory of $U\mathfrak{sl}_2$}
\label{subsec A Diagrammatic Calculus for sl2 Representations}


There is a diagrammatic way of describing intertwining maps between tensor powers of the fundamental representation $V_1$ of $U\mathfrak{sl}_2$. This was originally invented by Roger Penrose in his \emph{spin networks}, and later generalized to a $q$-deformed recoupling theory by Louis Kauffman. We begin with two simple maps:
\[
	\cup : \mathbb C \rightarrow V_1 \otimes V_1
\]
\[
	\cap : V_1 \otimes V_1 \rightarrow \mathbb C 
\]
defined by 
\[
	\cup(1) = \sqrt{-1} (e_0 \otimes e_1 - e_1 \otimes e_0)
\]
\[
	\cap(e_0 \otimes e_0) = \cap(e_1 \otimes e_1) = 0
\]
\[
	\cap(e_0 \otimes e_1) = \sqrt{-1} = -\cap(e_1 \otimes e_0)
\]
We also define $| : V_1 \rightarrow V_1$ to be the identity, and $\SingularCrossing{1} : V_1 \otimes V_1 \rightarrow V_1 \otimes V_1$ to be the flip map $\SingularCrossing{1}(a \otimes b) = b \otimes a$. 

\begin{lem}
\label{lem sl2 intertwiner properties}
The maps $\cup,\cap,|,\SingularCrossing{1}$ are all intertwiners of the $U\mathfrak{sl}_2$ action on $V_1$, and they satisfy the following properties:
\begin{enumerate}
	\item(Fundamental Binor Identity) $\SingularCrossing{1} = | \otimes | + \cup \circ \cap$
	\item(Zig-Zag Identity) $(| \otimes \cap) \circ (\cup \otimes |) = | = (\cap \otimes |) \circ (| \otimes \cup)$
	\item $(\cap \otimes |) \circ (| \otimes \SingularCrossing{1}) = (| \otimes \cap) \circ (\SingularCrossing{1} \otimes |)$
	\item $(\SingularCrossing{1} \otimes |) \circ (| \otimes \cup) = (| \otimes \SingularCrossing{1}) \circ (\cup \otimes |)$
	\item(Circle Evaluation) $\cap \circ \cup$ is the linear map $\mathbb C \rightarrow \mathbb C$ given by $z \mapsto -2z$.
\end{enumerate}
\end{lem}
\begin{proof}
The proof of these statements is just routine computations.
\end{proof}

Consider an immersion of a 1-manifold $C$ (with or without boundary) into the manifold $\mathbb R \times [0,1]$, such that the image of $\partial C$ lies in $\mathbb R \times \lcb 0,1 \rcb$ (see \cref{fig Immersed 1-manifold}). If we take a generic immersion, then we can divide the interval $[0,1]$ into pieces $c_1=0 < c_1 < \cdots < c_{n-1} < c_n = 1$ such that the piece of $C$ lying in $\mathbb R \times (c_{i-1},c_i)$ looks like all vertical strands $| | \cdots | |$, or many vertical strands with a cup $|| \cdots |\cup| \cdots ||$, a cap $|| \cdots |\cap| \cdots||$ or a crossing $|| \cdots |\SingularCrossing{1}| \cdots ||$. We will consider such immersions only up to \emph{regular homotopy}. That is, we consider two immersions $\varphi_0,\varphi_1 : C \rightarrow \mathbb R \times [0,1]$ equivalent if there is a homotopy $\varphi_t$ between $\varphi_0$ and $\varphi_1$ such that $\varphi_t$ is an immersion for all $t$. Such homotopies are generated by an immersed version of the Reidemeister II move $\ImmersedReidemeisterTwoTwo{.7} \leftrightarrow \ImmersedReidemeisterTwoOne{.7}$ and Reidemeister III move $\ImmersedReidemeisterThreeOne{.7} \leftrightarrow \ImmersedReidemeisterThreeTwo{.7}$. Note that we do not have a Reidemeister I move $\ImmersedReidemeisterOneOne{.7} \leftrightarrow \ImmersedReidemeisterOneTwo{.7}$, but a consequence of the immersed Reidemeister II move is the following: $\ImmersedRibbonReidemeisterOneOne{.7} \leftrightarrow \ImmersedRibbonReidemeisterOneTwo{.7}$.

\begin{figure}[tb]
\centering
\begin{tikzpicture}
[
	scale=3,
	bd/.style={circle,draw,fill=black,inner sep=0pt,minimum size=1mm}
]

	\draw (-1,0) -- (1,0);
	\draw (-1,1) -- (1,1);
	\draw[dashed] (-1.25,0) -- (-1,0);
	\draw[dashed] (1,0) -- (1.25,0);
	\draw[dashed] (-1.25,1) -- (-1,1);
	\draw[dashed] (1,1) -- (1.25,1);
	
	\node[bd] (b1) at (-.5,0) {};
	\node[bd] (b2) at (0,0) {};
	\node[bd] (b3) at (.5,0) {};
	
	\node[bd] (t1) at (-.75,1) {};
	\node[bd] (t2) at (-.5,1) {};
	\node[bd] (t3) at (0,1) {};
	\node[bd] (t4) at (.25,1) {};
	\node[bd] (t5) at (.75,1) {};
	
	\draw (b1) -- (t2);
	\draw[out=90,in=270] (b2) to node {} (t5);
	\draw[out=90,in=270] (b3) to node {} (t3);
	\draw[out=270,in=270,distance=8mm] (t1) to node {} (t4);
	
	\node[ellipse,draw,inner sep=0pt,minimum width=7mm,minimum height=10mm] at (-.3,.4) {};
	
\end{tikzpicture}
\caption{An immersed 1-manifold in $\mathbb R \times [0,1]$}
\label{fig Immersed 1-manifold}
\end{figure}

To each immersion of a 1-manifold into $\mathbb R \times [0,1]$ we will associate an intertwiner of tensor powers of $V_1$. In particular, suppose a generic immersion as $n$ boundary points on $\mathbb R \times 0$ and $n$ boundary points on $\mathbb R \times 1$. Then we associate an intertwiner $V_1^{\otimes n} \rightarrow V_1^{\otimes m}$ in the following way. Divide the interval $[0,1]$ into pieces as before so that each slice of the immersion consists of a bunch of vertical lines and at most one cup, cap or crossing. To each such slice we associate the obvious intertwiners:
\[
\begin{array}{ccl}
	|\cdots| &\mapsto& \id \otimes \cdots \otimes \id \\
	|\cdots|\cup|\cdots| &\mapsto& \id \otimes \cdots \otimes \cup \otimes \cdots \otimes \id \\
	|\cdots|\cap|\cdots| &\mapsto& \id \otimes \cdots \otimes \cap \otimes \cdots \otimes \id \\
	|\cdots|\SingularCrossing{1}|\cdots| &\mapsto& \id \otimes \cdots \otimes \SingularCrossing{1} \otimes \cdots \otimes \id 
\end{array}
\]
By composing the maps from each slice we get an intertwiner $V_1^{\otimes n} \rightarrow V_1^{\otimes m}$, and this is how diagrams of immersions of 1-manifolds induce maps on tensor powers of $V_1$. We can also consider formal linear combinations of diagrams of immersions, and they will induce form linear combinations of intertwiners. This gives us a diagrammatic interpretation of the identities in \cref{lem sl2 intertwiner properties}. For example, the fundamental binor identity is simply
\[ \SingularCrossing{3} = \ZeroSmoothing{3} + \OneSmoothing{3} \]
and the zig-zag identity can be drawn as
\[ \ZigZagOne{1} \ \ = \ \ \tikz[baseline=(current bounding box.center)] \draw (0,0)--(0,1); \ \ = \ \ \ZigZagTwo{1} \]
Further, it is easy to see that the maps satisfy the immersed versions of the Reidemeister II and III moves
\[ \ImmersedReidemeisterTwoOne{1} = \ImmersedReidemeisterTwoTwo{1} \ \ \ \ \ \ \ \ImmersedReidemeisterThreeOne{1} = \ImmersedReidemeisterThreeTwo{1} \]
This shows that the induced map $V_1^{\otimes n} \rightarrow V_1^{\otimes m}$ depends only on the regular homotopy type of the immersion.














\newpage
\section{Quantum Recoupling Theory}
\label{sec Quantum Recoupling Theory}


\subsection{The Hopf Algebra $U_q\mathfrak{sl}_2$ and its Representations}
\label{subsec The Hopf Algebra Uqsl2 and its Representations}


\subsection{A Diagrammatic Calculus for the Representation Theory of $U_q\mathfrak{sl}_2$}
\label{subsec A Diagrammatic Calculus for the Representation Theory of Uqsl2}









\end{document}

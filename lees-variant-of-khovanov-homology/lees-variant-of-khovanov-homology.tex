

\input{"/Users/brandonwilliams/Documents/LaTeX Includes/hwpreamble.tex"}
\input{"/Users/brandonwilliams/Documents/LaTeX Includes/extrapackages.tex"}
\input{"/Users/brandonwilliams/Documents/LaTeX Includes/extracommands.tex"}




\begin{document}


\title{\Large Lee's Variant of Khovanov Homology}
\author{\large Brandon Williams}

\maketitle



\section{Construction of Khovanov Homology}

Let $V$ be the graded $\mathbb Q$-vector space generated by two elements $v_\pm$ with gradings $p(v_\pm) = \pm 1$. We define a $(1+1)$-TQFT $\mathcal A$ by defining it on the set of generators of cobordisms
\[
\mathcal A \left( \smallcircle \right) = V
\]
\[
\mathcal A\left( \smallpairofpants \right) = m : V \otimes V \longrightarrow V  \ \ \ \ \ \ \ \mathcal A \left( \smallupsidedownpairofpants \right) = \Delta : V \longrightarrow V \otimes V 
\]
\[
\mathcal A \left( \smallcap \right) = i : \mathbb Q \longrightarrow V  \ \ \ \ \ \ \ \mathcal A \left( \smallcup \right) = \epsilon : V \longrightarrow \mathbb Q
\]
where the maps $m,\Delta,\epsilon,i$ are given by
\begin{align}
\label{mdei}
\begin{array}{lcclcclcclcc}
	m(v_+ \otimes v_+)     = v_+ & & & \Delta(v_+) = v_+ \otimes v_- + v_- \otimes v_+ & & &  i(1) = v_+ & & & \epsilon(v_+) = 0 \\
	m(v_\pm \otimes v_\mp) = v_- & & &  \Delta(v_-) = v_- \otimes v_-                  & & &             & & & \epsilon(v_-) = 1 \\
	m(v_- \otimes v_-)     = 0
\end{array}
\end{align}
The maps $m$ and $\Delta$ are commutative and cocommutative, and $i$ and $\epsilon$ satisfy the relations of a unit and counit. It is also easy to see that
\begin{equation}
\label{frobenius condition}
\Delta \circ m = (m \otimes \id) \circ (\id \otimes \Delta) = (\id \otimes m) \circ (\Delta \otimes \id)
\end{equation}
so these maps make $V$ into a commutative and cocommutative Frobenius algebra, which implies that $\mathcal A$ is well-defined.

Let $L$ be a link diagram with $k$ crossings, and fix a labeling of the crossings by numbers 1 through $k$. Each vector $v \in [0,1]^k$ consisting of only 1's and 0's (i.e. vertices of the $k$-dimensional cube) determine a resolution of $L$ by resolving the $i$-th crossing according to the $i$-th component of $v$ (see Figure~\ref{crossing resolution}). Let $L_v$ denote the associated resolution of $L$, which is just some disjoint collection of circles.
\begin{figure}[t]
\centering
\[
\xy0;/r0.5pc/:
(-6,6)*{}="tl";
(6,6)*{}="tr";
(-6,-6)*{}="bl";
(6,-6)*{}="br";
"tl";"tr" **\crv{(0,1)};
"bl";"br" **\crv{(0,-1)};
(0,-10)*{\text{0-resolution}};
\endxy
\qquad \qquad
\xy0;/r0.5pc/:
(6,6)*{}="tl";
(-6,6)*{}="tr";
(6,-6)*{}="bl";
(-6,-6)*{}="br";
{\ar@{-}|{\hole \; \hole \; \hole \; \hole \; \hole \; \hole } "bl";"tr"};
{\ar@{-} "br";"tl"};
(0,-10)*{\text{crossing}};
\endxy
\qquad \qquad
\xy0;/r0.5pc/:
(-6,6)*{}="tl";
(6,6)*{}="tr";
(-6,-6)*{}="bl";
(6,-6)*{}="br";
"bl";"tl" **\crv{(-1,0)};
"br";"tr" **\crv{(1,0)};
(0,-10)*{\text{1-resolution}};
\endxy 
\]
\caption{Ways to resolve a crossing}
\label{crossing resolution}
\end{figure}

If two vertices $v$ and $w$ of the cube $[0,1]^k$ share an edge, then their coordinates differ in precisely one place, say the $i$-th coordinate, and we can suppose $v_i=0$ and $w_i=1$. In this case we will say the edge of $[0,1]^k$ which contains these vertices is directed from $v$ to $w$. If $e$ is the edge from $v$ to $w$, we will let $C_e : L_v \rightarrow L_w$ denote the standard cobordism by taking the identity cobordism outside some small neighborhood of $i$-th crossing and taking the appropriate saddle cobordism from the 0-resolution to the 1-resolution around the $i$-th crossing (see Figure~\ref{saddle cobordism}).

\begin{figure}[t]
\centering
\[ \saddlecobordism \]
\caption{Saddle cobordism between 0- and 1-resolutions}
\label{saddle cobordism}
\end{figure}

We can now define the Khovanov complex of the link $L$ as the vector space
\[ \CKh(L) = \bigoplus_{v \in \lcb 0,1 \rcb^k} \mathcal A(L_v) \]
There is another way of interpreting $\CKh(L)$. Let us write $V = \mathbb Q v_+ \oplus \mathbb Q v_+$, then $V^{\otimes n}$ is the direct sum of all vector spaces of the form $\mathbb Q v_1 \otimes \cdots \otimes \mathbb Q v_n$ where $v_i = v_\pm$. On the other hand, given a resolution $L_v$ of $L$ consisting of $n$ circles, we can decorate each circle with either $v_i = v_+$ or $v_i = v_-$, and associate to this decoration the vector space $\mathbb Q v_1 \otimes \cdots \otimes \mathbb Q v_n$. Therefore $\CKh(L)$ can be thought of as the direct sum of the vector spaces associated to all possible decorations of all possible resolutions of $L$.

For a vector $v \in \lcb 0,1 \rcb^k$, let $c_0(v)$ and $c_1(v)$ denote the number of 0's and 1's in $v$ respectively. It is clear that there are exactly $c_0(v)$ edges of $[0,1]^k$ directed from $v$, so for each $1 \leq i \leq c_0(v)$ let $C_{e_i}$ denote the cobordism associated to the edge that changes the $i$-th zero in $v$ to a one. Then we can define the differential $d : \CKh(L) \rightarrow \CKh(L)$ for $x \in \mathcal A(L_v)$ by
\begin{equation}
\label{khovanov d}
d(x) = \sum_{i=1}^{c_0(v)} (-1)^{\sum_{j<i} v(i)} \mathcal A(C_{e_i})(x)
\end{equation}
In the exponent of $-1$ we are using the notation $v(i)$ to denote the $i$-th coordinate of $v$. 

\begin{prop}
\label{khovanov d^2 = 0}
$d^2 = 0$.
\end{prop}
%%%%%%%%%%%%%%%%%%%%%%%%%%%%%%%%%
%%%%%%%%%%%%%%%%%%%%%%%%%%%%%%%%%
%%%%%%%%%%%%%%%%%%%%%%%%%%%%%%%%%
\comment{
\begin{proof}
\comment{
Let $x \in \mathcal A(L_v)$ for some $v \in \lcb 0,1 \rcb^k$, and let $x_i = \mathcal A(C_{e_i})(x)$, where $C_{e_i}$ are the cobordisms discussed in the definition of \eqref{khovanov d}, so that $d(x) = \sum_i \pm \, x_i$. Each $x_i$ lies in $\mathcal A(L_{w_i})$, where $w_i$ is obtained from $v$ by changing the $i$-th zero to a one. Similarly $d(x_i) = \sum_j \pm \, y_{ij}$, where $y_{ij}$ lies in $\mathcal A(L_{z_{ij}})$ and $z_{ij}$ is obtained from $w_i$ by changing the $j$-th zero to a one. We now have that $d^2(x) = \sum_{i,j} \pm \, y_{ij}$. 

We claim that $y_{ij} = y_{ji}$. 

Finally we claim that $y_{ij}$ and $y_{ji}$ appear in $d^2(x)$ with opposite signs.

\todo{finish}
}
\end{proof}
}
%%%%%%%%%%%%%%%%%%%%%%%%%%%%%%%%%
%%%%%%%%%%%%%%%%%%%%%%%%%%%%%%%%%
%%%%%%%%%%%%%%%%%%%%%%%%%%%%%%%%%

The proof of \ref{khovanov d^2 = 0} is completely based on the fact that the maps $m$ and $\Delta$ are (co)associative and (co)commutative, and satisfied the Frobenius condition \eqref{frobenius condition}. If our TQFT $\mathcal A$ had been defined with other multiplication and comultiplication maps that satisfied these properties, we could still define a map $d$ and find that $d^2 = 0$.

We define the Khovanov homology vector space to be $\Kh(L) = \ker d / \image d$. Right now $\CKh$ is graded by elements of $\lcb 0,1 \rcb^k$, so we introduce some numerical degrees by defining the homological degree of $x \in \mathcal A(L_v)$ to be $\gr(x) = c_1(v) - n_-$, where $n_-$ is the number of negative crossings in the link diagram. 

\begin{prop}
\label{d raises hom degree}
The Khovanov differential raises the homological degree by one.
\end{prop}
\begin{proof}
In the definition of the differential \eqref{khovanov d} we see that $\mathcal A(C_{e_i})(x) \in \mathcal A(L_w)$, where $w$ is obtained from $v$ by changing the $i$-th zero to a one. So, $d(x)$ is a sum of elements of degree $c_1(w)-n_- = c_1(v)+1-n_- = \gr(x)+1$, hence $d$ raises the homological degree by one.
\end{proof}

Recall that $V$ is generated by two elements $v_\pm$ of degrees $p(v_\pm) = \pm 1$. We can extend this grading to $V^{\otimes n}$ by $p(v_1 \otimes \cdots \otimes v_n) = \sum p(v_i)$. 

\begin{prop}
\label{d lowers p degree}
If $e$ is an edge of $[0,1]^k$, $C_e$ the associated cobordism, and $v \in V^{\otimes n}$ is of homogeneous $p$-degree, then $p(\mathcal A(C_e)(v)) = p(v) - 1$.
\end{prop}
\begin{proof}
First we note that the maps $m,\Delta$ in \eqref{mdei} lower the $p$-grading by 1. Suppose $v = v_1 \otimes \cdots \otimes v_n$. Then for some $i$ we have that $\mathcal A(S_e)(v)$ is in two possible forms:
\[
\mathcal A(S_e)(v_1 \otimes \cdots \otimes v_n) = \begin{cases}
																										 v_1 \otimes \cdots \otimes v_{i-1} \otimes \Delta(v_i) \otimes v_{i+1} \otimes \cdots \otimes v_n \\
																										 v_1 \otimes \cdots \otimes m(v_i \otimes v_j) \otimes \cdots \otimes v_n 
																									\end{cases}
\]
In both cases the overall $p$-degree lowers by one since $\Delta$ and $m$ lower the degree by 1.
\end{proof}


There is another grading we can introduce in $\CKh(L)$. If $x \in \mathcal A(L_v)$ for some $v \in \lcb 0,1 \rcb^k$, then we define the quantum degree of $x$ by
\begin{equation}
\label{q grading}
q(x) = p(x) + \gr(x) + n_+ - n_-
\end{equation}
This makes $\CKh$ and $\Kh$ into a bigraded vector spaces. If we write $\CKh^{r,s}$ we will mean that $r$ is the homological grading and $s$ is the quantum grading, and if we write only one superscript $\CKh^r$ we will mean that $r$ is the homological grading.

\begin{cor}
\label{d preserves q}
The Khovanov differential preserves the $q$-grading, i.e. if $x \in \mathcal A(L_v)$, then $q(d(x)) = q(x)$.
\end{cor}
\begin{proof}
By \ref{d lowers p degree} $d$ lowers the $p$-grading, and by \ref{d raises hom degree} $d$ raises the homological grading, so it is clear from \eqref{q grading} that $d$ preserves the $q$-grading.
\end{proof}

Therefore $\CKh(L)$ splits as a direct sum of chain complexes.


\section{Lee's Spectral Sequence}


We will now define a new $(1+1)$-TQFT $\mathcal A'$, which in turn defines a new complex $\CKh'$ and homology vector space $\Kh'$. The new functor $\mathcal A'$ still maps a circle to $V$ and the disjoint union of circles to tensor products of $V$, and $\mathcal A'$ of a cup and cap are still given by $\epsilon$ and $i$ from \eqref{mdei}. However, we change our multiplication and comultiplication induced by pairs of pants by
\[
\mathcal A\left( \smallpairofpants \right) = m' : V \otimes V \longrightarrow V  \ \ \ \ \ \ \ \mathcal A \left( \smallupsidedownpairofpants \right) = \Delta' : V \longrightarrow V \otimes V 
\]
where
\begin{align}
\label{deformed md}
\begin{array}{lcclcclcclcc}
	m'(v_\pm \otimes v_\pm) = v_+ & & & \Delta'(v_+) = v_+ \otimes v_- + v_- \otimes v_+ \\
	m'(v_\pm \otimes v_\mp) = v_- & & & \Delta'(v_-) = v_- \otimes v_- + v_+ \otimes v_+
\end{array}
\end{align}
It is easy to see that these maps make $V$ into a commutative and cocommutative Frobenius algebra as before. Using $\mathcal A'$ we can define the vector space $\CKh'(L)$ and map $d' : \CKh'(L) \rightarrow \CKh'(L)$ in the same way as before
\[ CKh'(L) = \bigoplus_{v \in \lcb 0,1 \rcb^k} \mathcal A'(L_v) \]
\begin{equation}
\label{dprime}
d'(x) = \sum_{i=1}^{c_0(v)} (-1)^{\sum_{j<i} v(i)} \mathcal A'(C_{e_i})(x)
\end{equation}
Let $\Kh'(L)$ denote the homology. Since $\CKh'(L)$ is the same underlying vector spaces as $\CKh(L)$ we can define the homological grading and $q$-grading in the same way as before. Just as in \ref{d raises hom degree} we have that $d'$ raises the homological degree by one. Unfortunately $d'$ does not preserve the $q$-grading like last time (actually this turns out to be a very useful outcome). 

\begin{prop}
\label{dprime respects filtration}
Let $x \in \CKh'$ be of homogeneous $q$-grading, then each term of \eqref{dprime} in $d'(x)$ has $q$-grading greater than or equal to $q(x)$.
\end{prop}
\begin{proof}
First let us take note of the behavior of $p$-gradings under $m'$ and $\Delta'$ in \eqref{deformed md}. By examination of the $p$-gradings of elements we are plugging into $m'$ and $\Delta'$, and the terms in their outputs we see that $m'$ and $\Delta'$ lowers the $p$-grading of elements by at most one.

Suppose $x = v_1 \otimes \cdots \otimes v_n \in \mathcal A'(L_v)$, where $v_i = v_\pm$. Then $d'(x)$ is a sum of terms of the form
\begin{align*}
&v_1 \otimes \cdots \otimes m'(v_i \otimes v_j) \otimes \cdots \otimes v_n \\
&v_1 \otimes \cdots \otimes \Delta'(v_i) \otimes \cdots \otimes v_n 
\end{align*}
So it suffices to look at the special case of $x = v_1 \otimes v_2$. We have the following formulas for $q$-gradings
\begin{align}
\label{q-grading line 1}
q(v_1 \otimes v_2) &= p(v_1 \otimes v_2) + c_1(v) + n_+ - 2n_- \\
\label{q-grading line 2}
q(m'(v_1 \otimes v_2)) &= p(m'(v_1 \otimes v_2)) + c_1(v) + 1 + n_+ - 2n_- 
\end{align}
In \eqref{q-grading line 2} we see that $p(m'(v_1\otimes v_2))$ is at most one less than $q(v_1\otimes v_2)$, but then we have added one to $c_1(v)$ since we have increased the number of 1-resolutions. Therefore the difference between \eqref{q-grading line 2} and \eqref{q-grading line 1} is greater than or equal to zero. The same argument holds for $\Delta'$, therefore each term in $d'(x)$ has $q$-grading greater than or equal to $q(x)$.
\end{proof}

That fact that $d'$ either raises the $q$-grading of elements or preserves the $q$-grading allows us to define a filtration on $\CKh'$ that is respected by $d'$. We define a filtration $F$ on $\CKh'$ by
\begin{equation}
\label{filtration}
F^s \CKh' = \bigoplus_{v \in \lcb 0,1 \rcb^k} \lcb x \in \mathcal A'(L_v) \st q(x) \geq s \rcb 
\end{equation}
Clearly this is a decreasing filtration, as $F^s \CKh' \subset F^{s-1} \CKh'$, and the filtration is finite and exhaustive. By \ref{dprime respects filtration} $d'$ respects the filtration, i.e. $d'(F^s \CKh') \subset F^s \CKh'$. The filtration also respects the homological grading, i.e. $F^s \CKh'^r \subset F^{s-1} \CKh'^r$.  A ``basic principle'' of spectral sequences tells us that a graded object equipped with a differential and a filtration (such that everything is compatible) leads to a spectral sequence, as spelled out in the following theorem:

\begin{thm}
\label{Lee's spectral sequence}
There is a spectral sequence with $E_0 = \CKh$ and $E_1 = \Kh$, and the spectral sequence converges to $\Kh'$.
\end{thm}
\begin{proof}
The fact that there is a spectral sequence converging to $\Kh'$ is apparent from the ``basic principle.'' We will identify the $E_0$ and $E_1$ pages of this spectral sequence. 

The $E_0$ page is simply the bigraded vector space associated to our filtered, graded object:
\begin{equation}
\label{associated bigraded vector space}
E_0^{r,s} = F^s \CKh'^r / F^{s+1} \CKh'^r
\end{equation}
Since we are dealing with vector spaces it is easy to see that
\begin{align*}
	F^s \CKh'^r / F^{s+1} \CKh'^r &= \bigoplus_v \lcb x \in \mathcal A(L_v) \st \gr(x)=r, q(x) = s \rcb \\
	                              &= \CKh'^{r,s} \\
	                              &= \CKh^{r,s}
\end{align*}
where the last equality follows since the underlying vector spaces for both complexes is the same. The map $d'$ induces a map on $E_0$ such that $d' : E_0^{r,s} \rightarrow E_0^{r+1,s}$, which can be thought of as the part of $d'$ that preserves the $q$-grading rather than raising it. Since $d'$ is expressed in terms of $m'$ and $\Delta'$, the part of $d'$ that preserves $q$-grading is determined by the parts of $m'$ and $\Delta'$ that preserves the $q$-grading. Let us determine what part of $m'$ preserves the $q$-grading. 


Suppose $v_1 \otimes v_2 \in \mathcal A(L_v)$, then 
\begin{align*}
q(v_1 \otimes v_2) &= p(v_1 \otimes v_2) + c_1(v) + n_+ - 2n_- \\
q(m'(v_1 \otimes v_2)) &= p(m'(v_1 \otimes v_2)) + c_1(v) + 1 + n_+ - 2n_- 
\end{align*}
The part of $m'$ and $\Delta'$ that preserves $q$-grading is the part that lowers the $p$-grading by 1. Upon examination of \eqref{deformed md} we see that these parts are precisely $m$ and $\Delta$, therefore the differential on the $E_0$ page is the Khovanov differential on $\CKh$, hence $E_1 = \Kh$.
\end{proof}


The convention we used for the associated bigraded vector space \eqref{associated bigraded vector space} is not the one usually used. It is more common to use
\[ E_0^{s,r} = F^s \CKh'^{r+s} / F^{s+1} \CKh'^{r+s} \]
Notice that the indices in $E_0$ have been switched, and we are using $r+s$ for the homological grading. With this convention the bidegree of $d_r$ is $(r,1-r)$. This convention is used because with the other convention many popular spectral sequences (like the Leray-Serre spectral sequence) will live in a half-wedge in the first quadrant, which is awkward. By making this index transformation we get a spectral sequence living in the full first quadrant for those spectral sequences. However, Khovanov homology can be supported in any region in the bigraded vector space, so we do not adopt this convention. With our indices we have that the bidegree of $d_r$ is $(1,r)$.

It turns out that pretty much the entire spectral sequence is an invariant of the link.

\begin{thm}
The $E_1$ and higher pages of the spectral sequence in \ref{Lee's spectral sequence} are invariants of the link $L$.
\end{thm}

The proof of this theorem constructs a map $\rho_i' : \CKh'(L) \rightarrow \CKh'(\tilde{L})$ for links $L$ and $\tilde{L}$ that are related by the $i$-th Reidemeister move. It is then shown that $\rho_i'$ induces a map on the spectral sequence which is an isomorphism on the $E_1$ page, and by principles of spectral sequence it follows that the induced maps on the higher pages are also isomorphisms.



\section{Rasmussen's $s$-invariant}

It turns out that Lee's homology is quite simple:

\begin{thm}
If $L$ is a link with $n$ component, then $\Kh'(L) \cong \mathbb Q^{2^n}$.
\end{thm}

Now let us apply Lee's homology and the associated spectral sequence to a knot $K$. The filtration \eqref{filtration} on $\CKh'$ induces a filtration on $\Kh'$, which we will also denote by $F$, and we will denote its associated graded vector space by $\Gr^{**} \Kh'$. Since $K$ has only one component we have that $\Kh' \cong \mathbb Q \oplus \mathbb Q$, and so also $\Gr^{**} \Kh' \cong \mathbb Q \oplus \mathbb Q$. Let $s_{\max}(K)$ and $s_{\min}(K)$ be the maximum and minimum indices $s$ such that $\Gr^{*s} \Kh' \neq 0$. One can prove that the two copies of $\mathbb Q$ in $\Kh'$ are situated in a very specific way.

\begin{prop}
For any knot $K$, $s_{\max}(K) = s_{\min}(K) + 2$.
\end{prop}

So, we define $s(K)$ to be $s_{\max}(K)-1 = s_{\min}(K)+1$. Since the spectral sequence is an invariant of the knot, we have that $s_{\max},s_{\min},s$ are invariants of the knot.


%define induced filtration on $\Kh'$ and use to define $s_\min$ and $s_\max$ .....








\end{document}